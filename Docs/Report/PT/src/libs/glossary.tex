\newglossaryentry{autoregressive integrated moving average}
{
    name={Autoregressive Integrated Moving Average},
    description={é um modelo de aprendizagem de média móvel integrado autoregressivo}
}

\newglossaryentry{seasonal autoregressive integrated moving average}
{
    name={Seasonal Autoregressive Integrated Moving Average},
    description={é um modelo de aprendizagem de média móvel integrado autoregressivo
    sazonal}
}

\newglossaryentry{autoregressive integrated moving average with exogenous variable}
{
    name={Autoregressive Integrated Moving Average with Exogenous Variable},
    description={é um modelo de aprendizagem de média móvel integrado autoregressivo
    com variáveis exógenas}
}

\newglossaryentry{seasonal autoregressive integrated moving average with exogenous
variable}
{
    name={Seasonal Autoregressive Integrated Moving Average with Exogenous Variable},
    description={é um modelo de aprendizagem de média móvel integrado autoregressivo
    sazonal com variáveis exógenas}
}

\newglossaryentry{dataset}
{
    name={dataset},
    description={(Conjunto de dados) é uma coleção de dados normalmente tabulados. Por
    cada elemento destacam-se várias características. Cada coluna representa uma
    variável particular. Cada linha corresponde a um determinado membro do conjunto de
    dados em questão. Cada valor é conhecido como um dado}
}

\newglossaryentry{python}
{
    name={Python},
    description={é uma linguagem de programação de alto nível, interpretada, de script,
    imperativa, orientada a objetos, funcional, de tipagem dinâmica e forte}
}

\newglossaryentry{inteligencia artificial}
{
    name={Inteligência Artificial},
    description={é a inteligência similar à humana exibida por mecanismos ou software,
    para além de também ser um campo de estudo académico}
}

\newglossaryentry{machine learning}
{
    name={Machine Learning},
    description={é um subcampo da Engenharia e da ciência da computação que evoluiu do
    estudo de reconhecimento de padrões e da teoria da aprendizagem computacional em
    \gls{inteligencia artificial}}
}

\newglossaryentry{google colab}
{
    name={Google Colab},
    description={é um ambiente de desenvolvimento com a linguagem \gls{python} que não
    requer configuração e é executado utilizando a \gls{google cloud}}
}

\newglossaryentry{google cloud}
{
    name={Google Cloud},
    description={é uma suíte de computação em nuvem oferecida pelo \gls{google}}
}

\newglossaryentry{google}
{
    name={Google},
    description={é uma empresa multinacional de serviços online e software dos
    Estados Unidos}
}

\newglossaryentry{variaveis exogenas}
{
    name={Variáveis Exógenas},
    description={são variáveis cujos valores são determinados fora do modelo e são
    impostas ao modelo (no caso do \acrshort{arima} serão utilizadas para ajudar a fazer
    as previsões)}
}

\newglossaryentry{script}
{
    name={script},
    description={é um programa de computador, normalmente executado com um interpretador}
}

\newglossaryentry{grid search}
{
    name={grid search},
    description={é o processo de coleção de dados para configurar os parâmetros ideais para
    um determinado modelo}
}

\newglossaryentry{roadmap}
{
    name={roadmap},
    description={é um recurso visual de alto nível que mapeia a evolução do produto/projeto
    ao longo do tempo}
}
