\documentclass[\main.tex]{subfiles}

\chapter{Contextualização e Motivação}

\section{Contextualização}
O presente documento descreve o trabalho realizado na análise comparativa de algoritmos de
aprendizagem com base em séries temporais. O trabalho foi proposto pelo Professor Fábio Silva
cuja motivação é desenvolver uma plataforma de seleção dos melhores algoritmos para um
determinado conjunto de dados em determinadas condições. O trabalho realizado pelo autor
decorre do desenvolvimento do projeto final da Licenciatura em Engenharia Informática da
Escola Superior de Tecnologia e Gestão do Politécnico do Porto.\par

\section{Objetivos}
Para a realização deste projeto, foram definidos alguns objetivos principais, sendo que
parte deles contêm subobjetivos. Com a finalidade de manter uma boa coesão e garantia de
qualidade do projeto, foram definidas as seguintes necessidades:
\begin{itemize}
    \item Estudo de modelos de séries temporais - visto que o projeto se trata da análise de
    algoritmo de aprendizagem com base em séries temporais, estes dados também têm que ser
    estudados e compreendidos de modo obter uma melhor assimilação do comportamento dos
    modelos e dos resultados obtidos;
    \item Encontro de melhores modelos para os \textit{\glspl{dataset}} pretendidos - de modo
    a concretizar este objetivo, é necessário o estudo de alguns métodos específicos,
    nomeadamente a pesquisa em grelha (\textit{\gls{grid search}});
    \item Estudo de vários algoritmos de aprendizagem: \textit{\acrshort{arima}},
    \textit{\acrshort{arimax}}, \textit{\acrshort{sarima}} e \textit{\acrshort{sarimax}}.
    Cada modelo deve ser estudado e compreendido de forma parametrizada para que a integração
    com qualquer conjunto de dados não cause nenhum problema;
    \item Desenvolvimento de uma plataforma de testes de modelos de séries temporais com os 
    algoritmos de aprendizagem estudados.
    \item Integração dos \textit{\glspl{script}} com o \textit{\gls{google colab}}.
\end{itemize}

\section{Resultados}
Com a realização deste projeto, espera-se o desenvolvimento de uma plataforma de testes de
modelos de séries temporais. A plataforma deve conseguir testar vários algoritmos sobre um
\textit{\gls{dataset}}. Deve também ser necessário realizar \textit{grid searches}, isto é,
executar todos os modelos com todas as configurações para descobrir melhores configurações
para um determinado \textit{\gls{dataset}}.\par
Para que a plataforma funcione sem qualquer tipo de falhas, todo o código desenvolvido deverá
ser adaptado para ser executado em ambiente de \textit{\gls{google colab}}., de modo a
facilitar a manutenção do ambiente de trabalho e a fim de garantir um bom desempenho da
máquina no decorrer dos testes.\par

% \vspace{10pt}

% \begin{lstlisting}[language=Python, caption=Função de transformação de datas]
% def parser(x: int):
%     return datetime.strptime(f"190{x}", "%Y-%m")
% \end{lstlisting}

% \begin{lstlisting}[language=Python, caption=Função de transformação de datas]
% def parser(x):
%     return datetime.strptime(x, "%d/%m/%Y")
% \end{lstlisting}\par



\section{Estrutura}
O presente documento está dividido em 5 capítulo principais:
\begin{enumerate}
    \item Contextualização e Motivação - este capítulo pretende apresentar de forma sucinta
    e objetiva as circunstâncias a que surgiu o projeto, contendo o contexto, objetivos e
    resultados do mesmo. Neste capítulo está contida ainda uma parte destinada à fundamentação
    teórica do projeto, dando a conhecer, de uma forma um pouco abrangente, os principais
    tópicos.
    \item Concetualização do Problema - nesta secção estão detalhados os aspetos mais técnicos
    do problema, visto que contém a definição dos requisitos e a definição de uma arquitetura
    concetual.
    \item Metodologia de Operacionalização do Trabalho - neste capítulo é explorada e a
    metodologia de trabalho utilizada. Contém também, no final, uma descrição sobre aspetos
    mais técnicos do projeto.
    \item Discussão dos Resultados - é nesta secção que serão discutidos os resultados
    obtidos e a explicação de possíveis controversias.
    \item Conclusão - este capítulo visa realizar uma conclusão global do projeto e, por fim,
    uma descrição de possíveis melhoramentos futuros no trabalho desenvolvido.
\end{enumerate}\par

\newpage