\documentclass[\main.tex]{subfiles}

\chapter{Contextualização e Motivação}
\begin{singlespace}
\minitoc
\end{singlespace}
\vspace{20pt}

\section{Introdução}

\subsection{Contextualização}
O presente documento descreve o trabalho realizado na análise comparativa de algoritmos de
aprendizagem com base em séries temporais. O trabalho foi proposto pelo Professor Fábio Silva
cuja motivação é desenvolver uma plataforma de seleção dos melhores algoritmos para um
determinado conjunto de dados em determinadas condições. O trabalho realizado pelo autor
decorre do desenvolvimento do projeto final da Licenciatura em Engenharia Informática da
Escola Superior de Tecnologia e Gestão do Politécnico do Porto.\par

\subsection{Objetivos}
Este projeto foi focado no estudo de um modelo de aprendizagem (\acrshort{arima}) e as suas
variantes, com o objetivo de testar vários \textit{\glspl{dataset}} e desenvolver uma
plataforma que informe quais os melhores modelos e as melhores configurações dos mesmo para
cada \textit{\gls{dataset}}. Posto isto, cada modelo teve de ser estudado e compreendido de
forma parametrizada para que a integração com qualquer conjunto de dados não cause nenhum
problema.\par

\subsection{Resultados}
«Resultados obtidos»\par

\subsection{Estrutura}
O presente documento está dividido em 5 capítulo principais:
\begin{enumerate}
    \item Contextualização e Motivação - este capítulo pretende apresentar de forma sucinta
    e objetiva as circunstâncias a que surgiu o projeto, contendo o contexto, objetivos e
    resultados do mesmo. Neste capítulo está contida ainda uma parte destinada à fundamentação
    teórica do projeto, dando a conhecer, de uma forma um pouco abrangente, os principais
    tópicos.
    \item Concetualização do Problema - nesta secção estão detalhados os aspetos mais técnicos
    do problema, visto que contém a definição dos requisitos e a definição de uma arquitetura
    concetual.
    \item Metodologia de Operacionalização do Trabalho - neste capítulo é explorada e a
    metodologia de trabalho utilizada. Contém também, no final, uma descrição sobre aspetos
    mais técnicos do projeto.
    \item Discussão dos Resultados - é nesta secção que serão discutidos os resultados
    obtidos e a explicação de possíveis controversias.
    \item Conclusão - este capítulo visa realizar uma conclusão global do projeto e, por fim,
    uma descrição de possíveis melhoramentos futuros no trabalho desenvolvido.
\end{enumerate}\par
\vspace{5pt}

\section{Fundamentação Teórica}

\subsection{Série Temporal}
Tal como é abordado em temas como a estatística, economia e matemática aplicada, uma série
temporal é uma coleção de observações feitas sequencialmente ao longo do tempo. Em modelos
de regressão linear a ordem das observações é irrelevante, mas em séries temporais a ordem
dos dados é fundamental.\\
\indent A análise de séries temporais compreende métodos para analisar os dados, a fim de
extrair estatísticas significativas e outras características. Para fazer uma previsão
com uma série temporal é utilizado um modelo para prever valores futuros com base em
valores observados anteriormente.\\
\indent Existem 4 componentes que uma série temporal pode ter: i) Nível: valor base da série
se fosse uma linha reta; ii) Tendência (opcional): comportamento, normalmente linear,
crescente ou decrescente ao longo do tempo; iii) Sazonalidade (opcional): padrões repetitivos
ou ciclos de comportamento ao longo do tempo; iv) Ruído (opcional): variações nas observações
que não podem ser explicadas pelo modelo.\\
\indent Para concluir, todas as séries temporais têm nível e a maior parte também tem ruído.
No entanto, a tendência e a sazonalidade são ocasionais.\par

\subsection{Grid Search}
\textit{Grid search} (ou pesquisa em grelha) é o processo de coleção de dados para
configurar os parâmetros ideais para um determinado modelo. Dependendo do tipo de modelo
utilizado, alguns parâmetros são necessários. Esta pesquisa não se aplica apenas a um
tipo de modelo, ela pode ser aplicada ao modelo de aprendizagem para calcular os melhores
parâmetros a serem usados para qualquer modelo.\\
\indent Um ponto importante a sublinhar é que esta pesquisa pode ser extremamente cara em
termos computacionais e pode levar muito tempo até obter resultados. O \textit{grid search}
construirá um modelo em cada combinação de parâmetros possível. Ele itera por meio de cada
combinação de parâmetros e armazena um modelo para cada combinação.\par

\subsection{Modelo ARIMA}
O modelo de Média Móvel Integrada Autoregressiva - \textit{\acrfull{arima}} - foi o modelo
utilizado para realizar este trabalho, sendo que, todo o projeto foi abordado em torno
deste modelo e variações do mesmo. \textit{\acrshort{arima}} significa:
\begin{itemize}
    \item AR: "\textit{Autoregression}" (Autoregressão) - Um modelo que usa a relação entre
    uma observação e um número de observações atrasadas.
    \item I: "\textit{Integrated}" (Integrado) - O uso de diferenciação de observações (por
    exemplo, retirar uma observação de uma observação, no passo anterior) de forma a manter
    a série temporal estacionária.
    \item MA: "\textit{Moving Average}" (Média Móvel) - Um modelo que usa a dependência
    entre uma observação e o erro residual de um modelo de média móvel aplicado a
    observações atrasadas.
\end{itemize}\par
\indent Cada um destes componentes está especificado no modelo como um parâmetro. A notação
utilizada é \textit{\acrshort{arima}}(p, d, q):
\begin{itemize}
    \item p: Ordem de atraso - número de observações atrasadas incluídas no modelo.
    \item d: Grau de diferenciação - número de vezes que observações brutas são
    diferenciadas.
    \item q: Ordem da média móvel - tamanho da janela de media móvel
\end{itemize}

\newpage